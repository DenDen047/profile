%%%%%%%%%%%%%%%%%%%%%%%%%%%%%%%%%%%%%%%%%
% Medium Length Professional CV
% LaTeX Template
% Version 2.0 (8/5/13)
%
% This template has been downloaded from:
% http://www.LaTeXTemplates.com
%
% Original author:
% Trey Hunner (http://www.treyhunner.com/)
%
% Important note:
% This template requires the resume.cls file to be in the same directory as the
% .tex file. The resume.cls file provides the resume style used for structuring the
% document.
%
%%%%%%%%%%%%%%%%%%%%%%%%%%%%%%%%%%%%%%%%%

%----------------------------------------------------------------------------------------
%	PACKAGES AND OTHER DOCUMENT CONFIGURATIONS
%----------------------------------------------------------------------------------------

\documentclass{resume} % Use the custom resume.cls style

% コメント埋め込み用マクロ
\usepackage{color}
\newcommand{\todo}[1]{\textsf{\textcolor{red}{\textbf{TODO:} \textit{#1}}}}


% データセット名マクロ
\usepackage{xspace}

% 省略形
\newcommand{\etal}{\textit{et al}.}
\newcommand{\ie}{\textit{i}.\textit{e}.}
\newcommand{\eg}{\textit{e}.\textit{g}.}
	% マクロを読み込む

\usepackage[left=0.75in,top=0.6in,right=0.75in,bottom=0.6in]{geometry} % Document margins
\usepackage[dvipdfmx]{hyperref}
\newcommand{\tab}[1]{\hspace{.2667\textwidth}\rlap{#1}}
\newcommand{\itab}[1]{\hspace{0em}\rlap{#1}}
\name{Naoya Muramatsu}	% Your name
\address{\href{http://www.u.tsukuba.ac.jp/~s1411453/}{http://www.u.tsukuba.ac.jp/~s1411453/}}
\address{sh.mn.nat@gmail.com}	% Your phone number and email
\address{1-2 Kasuga, Tsukuba, Ibaraki Pref., Japan}	% Your address
% \address{123 Pleasant Lane \\ City, State 12345} % Your secondary address (optional)


\begin{document}

%----------------------------------------------------------------------------------------
%	EDUCATION SECTION
%----------------------------------------------------------------------------------------

\begin{rSection}{Education}

{\bf Cornell University} \hfill {August 2018 - }
\\ Will be a PhD candidate in Information Science

{\bf University of Tsukuba} \hfill {April 2014 - March 2018}
\\ Bachelor of Science in Media Sciences and Engineering
\\ College of Media Arts, Science and Technology
%Minor in Linguistics \smallskip \\
%Member of Eta Kappa Nu \\
%Member of Upsilon Pi Epsilon \\


\end{rSection}

%----------------------------------------------------------------------------------------
%	WORK EXPERIENCE SECTION
%----------------------------------------------------------------------------------------

\begin{rSection}{Research Experience}

    \begin{rSubsection}{National Institute of Technology, Nagano College}{April 2015 - March 2016}{Undergraduate Research}{}
    \item \todo{IoT機器を使った屋内位置情報に関する研究をしていた}
    \end{rSubsection}

    \begin{rSubsection}{University of Tsukuba}{April 2016 - March 2018}{Undergraduate Research}{}
    \item \todo{ECサイトのレビュー解析}
    \item \todo{Sonoliards}
    \item \todo{DeepHolo}
    \item \todo{DeepWear}
    \end{rSubsection}

    \begin{rSubsection}{University of Tsukuba}{April 2018 - March 2019}{Master Research}{}
    \item \todo{ロボットの話}
    \end{rSubsection}

    % \begin{rSubsection}{IIT Kanpur}{January 2015 - April 2015}{Manufacturing Process Project - Dragon Model}{}
    % \item Worked in a team of six people and came up with a model of Dragon with movable wings
    % \item Designed and fabricated a skeleton model of dragon with movable wings from scratch in lab employing processes of welding, brazing and casting
    % \item Received Certificate of Appreciation among 40 projects for its artwork and detailing
    % \end{rSubsection}

\end{rSection}

%----------------------------------------------------------------------------------------
%	TECHNICAL STRENGTHS SECTION
%----------------------------------------------------------------------------------------

\begin{rSection}{Technical Strengths}

    \begin{tabular}{ @{} >{\bfseries}l @{\hspace{6ex}} l }
    Programming     &  Python, C Language, C++, Ruby, LaTex, JavaScript, SQL \\
    Software        & Docker, Autodesk Fusion360, Processing, \\
                    & mbed \\
    Hardware        & Arduino, mbed, PhantomX AX Metal Hexapod \\
    \end{tabular}

\end{rSection}

%	EXAMPLE SECTION
%----------------------------------------------------------------------------------------

\begin{rSection}{Publications} \itemsep 4pt
%\item Ranked in National Top 0.2\% (amongst 1,200,000 candidates) in JEE Mains 2013 and Top 1\% (amongst 150,000 candidates) in IIT-JEE Advanced 2013
%\item Ranked in the State-wise Top 1\% (amongst 70,000 candidates) in State level Engineering competitive Exam (MP PET)
%\item Stood first in MBD Talent Search Exam conducted by state government, competing against more than 1000 participants

\item \textbf{Naoya Muramatsu}, Tetsuji Satoh, Takayasu Fushimi. 2017. \todo{Product Attribute Extraction Method Based on Transition Pattern of Review Point of View}. In \textit{Data Engineering and Information Management 2017} (DEIM '17). \textbf{Student Presentation Award}.  % レビュー観点の推移パターンに基づく商品属性の抽出手法
\item \textbf{Naoya Muramatsu}, Ooi Chun Wei, Takashi Miyazaki. 2017. Development of High Performance Filter for Indoor Positioning System. In \textit{ICISIP 2017 Oral}.
\item \textbf{Naoya Muramatsu}, Chun Wei Ooi, Yuta Itoh, and Yoichi Ochiai. 2017. DeepHolo: Recognizing 3D Objects using a Binary-weighted Computer-Generated Hologram. In \textit{SIGGRAPH Asia 2017 Posters} (SA '17), November 27– 30, 2017, Bangkok, Thailand. ACM, New York, NY, USA, 2 pages. DOI: https://doi.org/10.1145/3145690.3145725 (to appear)
\item \textbf{Naoya Muramatsu}, Kazuki Ohshima, Ryota Kawamura, Ooi Chun Wei, Yuta Sato, and Yoichi Ochiai. 2017. Sonoliards: Rendering Audible Sound Spots by Reflecting the Ultrasound Beams. In \textit{Adjunct Publication of the 30th Annual ACM Symposium on User Interface Software and Technology} (UIST ’17). ACM, New York, NY, USA, 57-59. DOI: https://doi.org/10.1145/3131785.3131807
\item Chun Wei Ooi, \textbf{Naoya Muramatsu}, and Yoichi Ochiai. 2018. Eholo glass: Electroholography glass. A lensless approach to holographic augmented reality near-eye display. In \textit{SIGGRAPH Asia 2018 Technical Briefs} (SA ’18), December 4–7, 2018, Tokyo, Japan. ACM, New York, NY, USA, 4 pages. DOI: https://doi.org/10.1145/3283254.3283288
\item Natsumi Kato, Hiroyuki Osone, Daitetsu Sato, \textbf{Naoya Muramatsu}, and Yoichi Ochiai. 2017. Crowd Sourcing Clothes Design Directed by Adversarial Neural Networks. In \textit{NIPS 2017 Workshop} (NIPS ’17).
\item Natsumi Kato*, Hiroyuki Osone*, Daitetsu Sato, \textbf{Naoya Muramatsu}, and Yoichi Ochiai. 2018. DeepWear: a Case Study of Collaborative Design between Human and Artificial Intelligence. In \textit{Proceedings of the Twelfth International Conference on Tangible, Embedded, and Embodied Interaction} (TEI ’18). ACM, New York, NY, USA, 529-536. DOI: https://doi.org/10.1145/3173225.3173302 (* Joint first authorship.)
\item Mose Sakashita, Yuta Sato, Ayaka Ebisu, Keisuke Kawahara, Satoshi Hashizume, \textbf{Naoya Muramatsu}, Yoichi Ochiai. 2017. Haptic Marionette: Wrist Control Technology Combined with Electrical Muscle Stimulation and Hanger Reflex. In \textit{SIGGRAPH Asia 2017 Posters} (SA '17). ACM, New York, NY, USA, Article 33, 2 pages. DOI: https://doi.org/10.1145/3145690.3145743

% \item \textbf{Mose Sakashita}, Yuta Sato, Ayaka Ebisu, Keisuke Kawahara, Satoshi Hashizume, Naoya Muramatsu, and Yoichi Ochiai. 2017. Haptic marionette: wrist control technology combined with electrical muscle stimulation and hanger reflex.
% \item \textbf{Mose Sakashita}, Tatsuya Minagawa, Amy Koike, Ippei Suzuki, Keisuke Kawahara, and Yoichi Ochiai. 2017. You as a Puppet: Evaluation of Telepresence User Interface for Puppetry. In Proceedings of the 30th Annual ACM Symposium on User Interface Software and Technology (UIST ’17). ACM, New York, NY, USA, 217-228. DOI: https://doi.org/10.1145/3126594.3126608
% \item \textbf{Mose Sakashita}, Kenta Suzuki, Keisuke Kawahara, Kazuki Takazawa, and Yoichi Ochiai. 2017. Materialization of motions: tangible representation of dance movements for learning and archiving. In ACM SIGGRAPH 2017 Studio (SIGGRAPH '17). ACM, New York, NY, USA, Article 7, 2 pages. DOI: https://doi.org/10.1145/3084863.3084869
% \item Amy Koike, Satoshi Hashizume, Kazuki Takazawa, \textbf{Mose Sakashita}, Daitetsu Sato, Keisuke Kawahara, and Yoichi Ochiai. 2017. Digital fabrication and manipulation method for underwater display and entertainment. In ACM SIGGRAPH 2017 Posters (SIGGRAPH '17). ACM, New York, NY, USA, Article 76, 2 pages. DOI: https://doi.org/10.1145/3102163.3102226
% \item Ayaka Ebisu, Satoshi Hashizume, Kenta Suzuki, Akira Ishii, \textbf{Mose Sakashita}, and Yoichi Ochiai. 2017. Stimulated percussions: method to control human for learning music by using electrical muscle stimulation. In Proceedings of the 8th Augmented Human International Conference (AH '17). ACM, New York, NY, USA, Article 33, 5 pages. DOI: https://doi.org/10.1145/3041164.3041202
% \item Amy Koike, Satoshi Hashizume, \textbf{Mose Sakashita}, Yuki Kimura, Daitetsu Sato, Keita Kanai, and Yoichi Ochiai. 2016. Syringe-worked mermaid: computational fabrication and stabilization method for cartesian diver. In SIGGRAPH ASIA 2016 Posters (SA '16). ACM, New York, NY, USA, Article 35 , 2 pages. DOI:https://doi.org/10.1145/3005274.3005316
% \item Ayaka Ebisu, Satoshi Hashizume, Kenta Suzuki, Akira Ishii, \textbf{Mose Sakashita}, and Yoichi Ochiai. 2016. Stimulated percussions: techniques for controlling human as percussive musical instrument by using electrical muscle stimulation. In SIGGRAPH ASIA 2016 Posters (SA '16). ACM, New York, NY, USA, , Article 37 , 2 pages. DOI: https://doi.org/10.1145/3005274.3005324
% \item Keisuke Kawahara, \textbf{Mose Sakashita}, Amy Koike, Ippei Suzuki, Kenta Suzuki, and Yoichi Ochiai. 2016. Transformed Human Presence for Puppetry. In Proceedings of the 13th International Conference on Advances in Computer Entertainment Technology (ACE2016). ACM, New York, NY, USA, Article 38 , 6 pages. DOI: https://doi.org/10.1145/3001773.3001813
% \item \textbf{Mose Sakashita}, Keisuke Kawahara, Amy Koike, Kenta Suzuki, Ippei Suzuki, and Yoichi Ochiai. 2016. Yadori: mask-type user interface for manipulation of puppets. In ACM SIGGRAPH 2016 Emerging Technologies (SIGGRAPH '16). ACM, New York, NY, USA, Article 23 , 1 pages. DOI: http://dx.doi.org/10.1145/2929464.2929478

\end{rSection}

%----------------------------------------------------------------------------------------
\begin{rSection}{Work Experience}

    \begin{rSubsection}{Fixstars Corporation}{August 2014 - Septenber 2014}{Software Engineer Intern}{}
    \item Worked on development of semantic segmentation system for self-driving cars.
    \end{rSubsection}

    \begin{rSubsection}{Fixstars Corporation}{August 2016 - December 2016}{Software Engineer Intern}{}
    \item Worked on development of semantic segmentation system for self-driving cars.
    \end{rSubsection}

    \begin{rSubsection}{Pixie Dust Technologies, Inc.}{August 2017 - Present}{Software Engineer}{}
    \item Worked on development of management systems and web applications.
    \end{rSubsection}

    \begin{rSubsection}{CyberAgent Tech Kids, Inc.}{March 2015 - October 2016}{Programming School Mentor}{}
    \item Gave children introductory programming lessons.
    \end{rSubsection}
\end{rSection}

%\begin{rSection}{POSITION OF RESPONSIBILITY}
%
%\begin{rSubsection}{Techkriti 2015 - Technical and entrepreneurial Festival }{August 2014 - March 2015}{Public Relations}{IIT Kanpur}
%\item Spearheaded a 2-tier team of 40 people to successfully conduct professional shows, exhibitions and talks
%\item Organized talks in Techkriti by eminent personalities like Dr K. Radhakrishnan (Chairman, ISRO), Peter Schultz (Co-inventor, Fibre optics) and David Hilmers (NASA Astronaut) with more than 1000 attendees
%\item Successfully organized Auto expo, Space expo and Defence expo together for the first time in Techkriti
%\item Promoted awareness through social campaigns like Make a wish, Adopt a tree and Teen Suicide Prevention
%\end{rSubsection}
%
%%------------------------------------------------
%
%\begin{rSubsection}{Students' Placement Office}{April 2015 - Present}{Internship Coordinator}{IIT Kanpur}
%\item Coordinating with team of 20 students responsible for facilitating internship proceedings of 650 students involving 150 companies
% \item Responsible for developing contacts with corporate recruitment teams of several firms for internship and placements
% \item Organized sessions on Personality Development and Career Awareness by esteemed alumni for over 1600 students
%\end{rSubsection}
%
%%------------------------------------------------
%
%\begin{rSubsection}{Hall Executive Committee }{April 2014 - Nov 2014}{Secretary}{IIT Kanpur}
%\item Coordinated with 12 members to led a team of 200 students in inter hall technical, cultural and sports competition of institute
%\item Planned an annual budget of ₹ 2 lakhs for proper functioning of hostel with more than 400 residents
%\end{rSubsection}
%
%\end{rSection}

%----------------------------------------------------------------------------------------
\begin{rSection}{awards and honors}
\begin{tabular}{ @{} >{\bfseries}l @{\hspace{4ex}} l }
2018- & \textbf{The Nakajima Foundation}. Graduate Research Fellowship for Japanese. \\
 & \textbf{}Including tuition up to JPY3,000,000 per year, and stipend of JPY200,000 per month.\\
2017 & Japan Student Service Organization (JASSO) Excellence Student Award, \textbf{First Prize}. \\
2017 & Asian CHI Symposium. \textbf{Best Demo/Poster Award 2nd Prize}. \\
2017 & Advancing Researcher Experience (ARE), University of Tsukuba, \textbf{Excellence Award}. \\
2016 & HackU Tsukuba 2015 Sponsored by Yahoo Japan, \textbf{Grand Prize}. \\

\end{tabular}

\end{rSection}

\end{document}
